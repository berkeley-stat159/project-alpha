\documentclass[11pt]{article}

\usepackage[margin=0.75in]{geometry}
\usepackage{indentfirst}
\usepackage{graphicx}
\bibliographystyle{siam}

\title{Appendix: Convolution Analysis}
\author{
  LeRoy, Benjamin\\
  \texttt{benjaminleroy}
  \and
  Liang, Jane\\
  \texttt{janewliang}
}

\begin{document}
\maketitle



\section{Introduction:}

FMRI data presents a unique challenge from the data that is collected, attempting to read a response to the desired effect. This reponse to stimulus, a hemodyamic reponse (change in oxgenation levels of hemoglobin flowing through the region) has been well analyzed.

Over the course of this project, our group has explored 5 different ways to change the stimulus to a predicted hemodynamic response.

\section{0. theoretical idea behind convolution}
\section{1. basic convolution idea (\texttt{np.convolve}), FFT}

\section{2. Initial correction to represent theoretical idea}

\section{3. More efficient approaches}

\subsection{Matrix multiplication}

\subsection{Utilization of FFT with \textt{np.convolve}}

\section{4. Inclusion of time correction, a different approach}

\section{5. Final thoughts:}
 - comment about intial approach with 3 different hrf for the different conditions (little interpretation, but could try again)




\bibliography{project}

\end{document}
