% tex file for regression results
\par To develop linear models, we looked at the HR from the neural response 
as a single feature. Originally we used multiple regression to take into 
account the 3 different types of stimulus (pump, explode, cash-out) to see if 
the separation of these stimuli can better describe the response, but it wasn't 
that good. In figure \ref{fig:all_cond_time}, we can see the different 
conditions broken up. Using smoothed data, fourier and drift features we got 
the following fitted values and residuals for a random voxel for subject 001. [
Figure \ref{fig:fit_vs_act}, \ref{fig:fit_vs_res}].

  
\begin{figure}[ht]
\centering
\begin{minipage}[b]{0.45\linewidth}
	\centering
	\includegraphics[width=.8\linewidth]{images/Fitted_v_Actual.png} 
	\caption{Fitted vs Actual}
	\label{fig:fit_vs_act}
\end{minipage}	
\quad
\begin{minipage}[b]{0.45\linewidth}
	\centering
		\includegraphics[width=.8\linewidth]{images/Fitted_v_Residuals.png} 
	\caption{Fitted vs Residual}
	\label{fig:fit_vs_res}
\end{minipage}
\end{figure}




\begin{figure}[ht]
\centering
\includegraphics[scale=.5]{images/all_cond_time}  
\caption{Plotting all predicted HR for conditions.}
\label{fig:all_cond_time}
\end{figure}

As we also obtained $\hat{\beta}$ values (coefficients) from the linear 
regression models, we looked at the 3-dimensional reports of the 
$\hat{\beta}$ values, a less rigorous analysis than hypothesis testing with 
t-statistics [Figure \ref{fig:con1_beta_brain}]. 

\begin{figure}[ht]
\centering
\includegraphics[scale=.5]{images/mr_cond1_beta_brain}    
\caption{$\hat{\beta}$ values for condition 1, subject 001.}
\label{fig:con1_beta_brain}
\end{figure}

The numerous other multiple regression models discussed in 
\textit{Linear Regression} should be analyzed similarly in the future. 
