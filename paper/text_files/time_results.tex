% tex file for time series results
\par \indent For the time being, we considered only a single voxel from the 
first subject. To specify the orders $p$, $d$, and $q$ for an ARIMA process, 
we first checked for stationarity by considering the mean function and 
autocovariance. Visual inspection suggested that the mean function was 
nonconstant, and additional histograms and quantile-quantile plots suggested 
a slight skewness with a long right tail. We corrected the skewness using a 
log transformation, deemed appropriate by the Box-Cox method. The log-
transformed data still exhibited a non-constant mean, so we considered using 
the first difference (i.e. $W_t = Y_t - Y_{t-1}$). The first difference 
appeared to be much more reasonably stationary. So, we took $d=1$. 

\par Having specified the order for $d$, we turned to the problem of 
specifying $p$ and $q$. We used a combination of visually inspecting the 
autocorrelation and partial autocorrelation plots of the first difference, 
and looking at the Akaike information criteria (AIC) and Bayesian information 
criteria (BIC) computed from a grid of possible models. The latter method 
suggested specifying $p=1$ and $q=1$ (based on either the AIC or the BIC), 
which was also supported by the visual inspections. 

\par We estimated the parameters for an ARIMA(1,1,1) model using the exact 
maximum likelihood estimator via Kalman filter. The residuals appear to be 
normally distributed, and its autocorrelation and partial autocorrelation 
plots also do not raise any red flags. Furthermore, when visually comparing 
the fitted time series to the true observed data, the ARIMA process seems to 
approximate the observed data much better than any of the linear regression 
models. While more work, such as developing more robust methods to assess fit 
and considering the problem of modeling multiple voxels across multiple 
subjects, is clearly needed, time series analysis presents a promising 
direction for further investigation. 

\par In particular, we may be able to forecast future observations based on 
previous ones. As an example we modeled an ARIMA(1,1,1) process based on the 
first half of the observations for a single voxel. This process was then used 
to forecast the second half of the observations. A comparison between the 
true observations and the forecasted predictions is shown in [Figure \ref
{fig:ts-preds}]. While the forecasted observations look reasonable for 
approximating the true values, more quantitative metrics for assessing 
performance need to be implemented. 

\begin{figure}[ht]
\centering
\includegraphics[scale=0.5]{images/ts-preds.png}
\caption{Forecasting the second half of observations based on the first 
half.}
\label{fig:ts-preds}
\end{figure}

